\documentclass{manuscript}

\usepackage{kantlipsum}

\usepackage{fontspec}
\setmainfont{Brill}
\setmonofont{Menlo}[Scale=MatchLowercase]

\usepackage{ktav}

% \renewfontfamily\hebfont{BenOr Rashi}

\title{Comparing the \heb{.tl} Biblical Hebrew}
\author{Jackson Petty}
\address{Department of Linguistcs\\Yale University\\New Haven, Conn.}
\email{research@jacksonpetty.org}

\begin{document}
  \maketitle
  Rabbinic Hebrew (\textsc{rh}) does not differ greatly from Biblical Hebrew (\textsc{bh}) in its inflection of the noun, although the neutralization of final \emph{mem} and \emph{nun} means that the masculine plural is often, as in Aramaic, \heb{-in}. Apart from the more frequent use of the archaic feminine suffix \heb{-as} as in \heb{kohenes} `priest's wife' and \heb{'illemes} `dumb woman', \textsc{rh} also employs the
  suffixes \heb{-is} and \heb{-oos} for example \heb{':ar+miys} `Aramaic' and 
  \heb{`av'doos} `servitude'. \textsc{rh} developed distinctive feminite plural suffixes
  in \heb{-au'Os} (Babylonian) or \heb{-auyOs} (Palestinian), for example \heb{mar'ch:atsau'Os/mar'ch:atsauyOs} `bath-houses' and \heb{-iyyOs}, as in 
  \heb{mal'khiyyOs} `kingdoms' for \textsc{bh} \heb{mal'khuyOs}, for nouns ending in \heb{-Os} in the singular. Masculine plural forms sometimes differ from those that would be expected, or are normally found, in \textsc{bh}, for example, \heb{n'ziqiyn} from \heb{neizeq} `damage', \heb{sh^'wauriym} from \heb{sh^Or} `ox', \heb{sh^'wauqim} from \heb{sh^ooq} `market', \heb{ts'daudiyq} from \heb{tsad} `side', \heb{ch:atsau'iyn} from \heb{ch:atsiy} `half', and \heb{sh^:loochiyn} from \heb{sh^auliycha} `envoy'. The same is true of feminine nouns, for example \heb{'OtiyyOs} from \heb{'Os} `letter (of alphabet)', \heb{b:riytOs} from \heb{b:riys} `covenant (without plural in \textsc{bh})', and \heb{'immauhOs} from \heb{'eim} `mother'.
  
  Some masculine nouns take the feminine plural suffix \heb{-Os}, for example, 
  \heb{chinOs} from \heb{chein} `favor', \heb{k:laulOs} from \heb{kh:laul} `rule',
  \heb{tynoqOs} from \heb{tinOq} `baby', \heb{ch:ayaulOs} from \heb{chayil} `army',
  \heb{`:ayaurOs} from \heb{`iyr} `city', and \heb{meiymOs} from \heb{mayim} `water'.
  Similarly, there are some feminine nouns which take the masculine plural suffix \heb{-iym}---\heb{yOniym} from \heb{yOnauh} `dove', \heb{n:mauliym} from \heb{n:maulauh} `ant', and \heb{beiytsiym} from \heb{beitsauh} `egg', for example. Occasionally, both
  types of plural are evidenced, as with \heb{yaumiym/yaumOm} from \heb{yOm} `day' or 
  \heb{sh^auniym/sh^aunOt} from \heb{sh^aunauh} `year', with each form having a
  slightly different shade of meaning and the `feminine' variant only used with suffixes.
  In \textsc{rh} we sometimes find plurals of nouns only attested in the singular in \textsc{bh}, for example \heb{':avauriyt} from \heb{'eiver} `limb', \heb{dd:sh^au'iyn}
  from \heb{ddesh^e'} `grass', and \heb{t:midiym} from \heb{taumiyd} `daily sacrifice'.
  Likewise, there are singular forms of nouns only attested in the plural in \textsc{bh},
  form example \heb{'il:moog} `coral-wood', \heb{beiytsauh} `egg', and \heb{bautsaul} `onion'. The dual is used more than in \textsc{bh}, with existing forms retained and
  new ones created, for example \heb{ma.s:paurayim} `scissors' and \heb{bein:tayim} `meanwhile'. (1993: A.\ S'aenz-Badillos, \emph{A History of the Hebrew Language}, Cambridge University Press pp.\ 188--89.)
\end{document}